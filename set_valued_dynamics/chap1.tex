\documentclass{article}

\title{Chapter 1 Exercises}
\usepackage{dztex}
\usepackage{cleveref}
\usepackage{xspace}

\fancyhead[L]{Set-Valued, Convex and Nonsmooth Analysis in Dynamics and Control \vspace{1mm}}

\newenvironment{ex}[1]
  {\renewcommand\theexercise{#1}\exercise}
  {\endexercise}

\newcommand{\ud}{\,\mathrm{d}}
\newcommand{\cara}{Carath\'eodry\xspace}
\newcommand{\kras}{Krasovskii\xspace}
\newcommand{\flip}{Flippov\xspace}
\newcommand{\conc}{\mathrm{\overline{con}}\,}

\begin{document}
\begin{ex}{1.2}
  Starting with (a), due to the Maximum Principle
  $$
  u(t) = \begin{cases} -1 & t < \tau \\ 1 & t \ge \tau \end{cases}
  $$
  since the vehicle needs to initial accelerate towards the wall, for some $\tau \in [0, T]$. We can do a bit of analysis with this characterization of $u$:
  $$
    \ddot{z}(t) = u(t) \implies \dot{z}(t) = \begin{cases} 
      -t + \dot{z}(0) & t < \tau \\
      t - 2\tau + \dot{z}(0) & t \ge \tau
    \end{cases}, z(t) = \begin{cases}
      -\frac{t^2}{2} + \dot{z}(0)t + z(0) & t < \tau \\
      \frac{t^2}{2} - 2\tau t + \dot{z}(0) t + \tau^2 + z(0) & t \ge \tau
    \end{cases}
  $$
  We also know that $\dot{z}(T) = z(T) = 0$-- we can leverage that to solve for both $\tau, T$ by first analyzing the velocity equation:
  $$
  \dot{z}(T) = 0 \implies T = 2\tau - \dot{z}(0)
  $$
  And now similarly the acceleration equation:
  $$
  \ddot{z}(T) = 0 \implies 0 = \frac{T^2}{2} - 2\tau T + \dot{z}(0) T + \tau^2 + z(0)
  $$
  Plugging in the the former equation for the latter we get:
  $$
  0 = \tau^2 - 2 \dot{z}(0) \tau - z(0) + \frac{\dot{z}(0)^2}{2}
  $$
  For (a) this reduces to $0 = \tau^2 - 2 \implies \tau = \sqrt{2}$ and so $T = 2\sqrt{2}$. For (b) we must consider the above scenario and the one where the sign of $u$ flips in each branch. In the flipped scenario our equations become
  $$
    \ddot{z}(t) = u(t) \implies \dot{z}(t) = \begin{cases} 
      t + \dot{z}(0) & t < \tau \\
      -t + 2\tau + \dot{z}(0) & t \ge \tau
    \end{cases}, z(t) = \begin{cases}
      \frac{t^2}{2} + \dot{z}(0)t + z(0) & t < \tau \\
      -\frac{t^2}{2} + 2\tau t + \dot{z}(0)t - \tau^2 + z(0) & t \ge \tau
    \end{cases}
  $$
  Similarly using $\dot{z}(T) = z(T) = 0$ we find the following:
  $$
  T = 2\tau + \dot{z}(0), \; 0 = -\frac{T^2}{2} + 2\tau T + \dot{z}(0)T - \tau^2 + z(0)
  $$
  and combining we get
  $$
  0 = \tau^2 + 2\dot{z}(0)\tau + \frac{\dot{z}(0)^2}{2} + z(0)
  $$
  So, for (b) we have $z(0) = 2, \dot{z}(0) = -2$ so using the first construction of $u$ gives
  $$
  \tau^2 + 4\tau - 2 + 2 = 0 \implies \tau = 0, T = 2
  $$
  and the second construction:
  $$
  \tau^2 - 4\tau + 4 = 0 \implies \tau = \frac{4 \pm \sqrt{16 - 16}}{2} = 2 \implies T = 2
  $$
  Both choices of $u$ minimize with $T = 2$, and in fact with the given $\tau$ the two constructions are identical, i.e. $u = 1$ with $T = 2$. Lastly for (c) we have $z(0) = 2, \dot{z}(0) = -4$. In the first construction of $u$ we would get
  $$
  \tau^2 + 8 \tau + 6 = 0 \implies \tau = \frac{-8 \pm \sqrt{64 - 24}}{2} = -4 \pm \sqrt{10} < 0
  $$
  which isn't valid, so our only possible minimizing $u$ is from the second construction, which would give us
  $$
  \tau^2 - 8\tau + 10 = 0 \implies \tau = \frac{8 \pm \sqrt{64 - 40}}{2} = 4 \pm \sqrt{5}, T = 8 \pm 2\sqrt{5} - 4 = 4 \pm 2\sqrt{5}
  $$
  Of the above choices only $T = 4 + 2\sqrt{5}$ is valid, thus it's the minimal time, with $\tau = 4 + \sqrt{5}$ specifying the minimizing control (from the second variant of $u$).
\end{ex}
\begin{ex}{1.5}
  With an initial condition of $(0, 1)$ then the solution lies entirely on the x-axis, so our differential equation reduces to
  $$
  \dot{x}_1 = x_1^2 u(x_1, t)
  $$
  For $x_1 \ge \delta$ then $x + e(t) \ge 0$ and so $u(x + e(t)) = -1$, thus the solution continues towards the origin. However once $x_1 \in (-\delta, \delta)$ then $u$ will flip between $1, -1$ depending on whether $t \in \bracp{-\frac{\pi}{2\omega}, \frac{\pi}{2\omega}} + \frac{2\pi n}{\omega}$ for some $n \in \mathbb{Z}$, so in particular the solution cannot converge to the origin as it will flip direction every $\frac{\pi n}{\omega}$ time units (i.e. its periodic).
\end{ex}
\begin{ex}{1.7}
  If $x \ge 1$ then notice $y > 0$ will be a minimizer, and so calculus tells us $y = x - 1$. When $x \in (0, 1)$ then $y = 0$ is a minimizer. Thus given a $x_0 > 0$ it will take $\floor{x_0} + 1$ steps to reach $0$. The same argument can be made about negative $x$, except when $x \le -1$ the minimizer is $y = x + 1$.
\end{ex}
\begin{ex}{1.8}
  For our solutions we consider a constant sequence of $\phi_i := \phi$, with $\phi$ described below:
  \begin{itemize}
    \item
      Fix $I = \braces{0, 1, 2}$ and $e(0) = 6.5, \phi(1) = 0.5, e(1) = 0, \phi(2) = 0$.
    \item
      Fix $I = \braces{0, 1, 2, 3}$ and $e(0) = 0, \phi(1) = 4, e(1) = 4.5, \phi(2) = 8.5, e(2) = 0, \phi(3) = 0$.
  \end{itemize}
\end{ex}
\begin{ex}{1.9}
  Classically if such $\phi, T$ exist then $\dot{\phi}(0)$ must exist and so for any $\epsilon > 0$ the mean value theorem tells us $\exists c \in (0, \epsilon)$ so that $\dot{\phi}(c) = \frac{\dot{\phi}(\epsilon)}{\epsilon}$, i.e. $\mathrm{sgn}\, \dot{\phi}(\epsilon) = \mathrm{sgn}\, \phi(\epsilon)$. W.l.o.g. consider $\mathrm{sgn}\,\phi(\epsilon) = 1$, then from the diff eq we know $\dot{\phi}(\epsilon) = -1$, a contradiction.

  Now suppose such a $\phi, T$ exist as a \cara solution. Take a points $c, d \in [0, T]$ s.t. $c < d$ and w.l.o.g. consider $0 \le \phi(c) < \phi(d)$ then $\dot{\phi}(c) = \phi(d) = -1$. By absolute continuity we have:
  $$
  \phi(d) - \phi(c) = \int_{c}^{d} \dot{\phi}(t) \ud t = \int_{c}^{d} -1 \ud t = - \delta
  $$
  but $\phi(d) > \phi(c)$. N.B. we know we can find an increasing segment of $\phi$ when $\phi$ is positive by continuity (and we would be able to find a decreasing segment when $\phi$ is negative to show the same contradiction).
\end{ex}
\begin{ex}{1.10}
  The convex regularization of $f$ contains the line segment between $(0, 1) -- (0, -1)$, thus $0 \in F(0)$ and so $\phi = 0$ is a \flip / \kras solution. For
  $$
  f(x, y) = \begin{cases}
    (2, 1) & y < 0 \\
    (5, 2) & y \ge 0
  \end{cases}
  $$
  $x$ is free in the sense that $\dot{\phi}_x$ can be chosen based off $\dot{\phi}_y$. Ergo, the differential equation simplifies to:
  $$
  y' = f_y(y) := \begin{cases}
    1 & y < 0 \\
    2 & y \ge 0
  \end{cases}
  $$
  Our \cara solutions look like:
  $$
  y_0 < 0 \implies \phi(t) = \begin{cases}
    t + y_0 & t < -y_0 \\
    2t + 2y_0 & t > -y_0
  \end{cases}, \; y_0 \ge 0 \implies \phi(t) = 2t + y_0
  $$
  Note, for $y_0 < 0$ $\phi(-y_0)$ can be anything, as $\dot{\phi}(-y_0)$ doesn't need to be defined (as per the definition of a \cara solution).

  \,\\

  The \kras regularization of our simplified differential equation is trivially $F_{K}(y) = \braces{f_y(y)}$ whenever $y \ne 0$. If $y_0 > 0$ then it's clear the regularization is can be identified directly $f_y$ (since the derivative is always positive), so the \kras, and therefore \flip solutions are the same as the \cara solutions. When $y_0 < 0$ the solutions are identical until $\phi(t) = 0 (t = -y_0)$. In effect the \cara solutions agree with \kras (and therefore \flip) solutions a.e..

  \,\\
  N.B. I'm not sure if this is enough, i.e. is agreeing almost everywhere the same as being "the same"?

  ---
  \,\\
  For
  $$
  f(x, y) = \begin{cases}
    (2, 1) & y < 0 \\
    (1, 0) & y = 0 \\
    (5, -2) & y > 0
  \end{cases}
  $$
  A \flip solution that's not \cara is
  $$
  \phi(t) = \begin{cases}
    (2t + 1, t -1) & t \in [0, 1] \\
    (3t, 0) & t \in (1, \infty)
  \end{cases}
  $$
  since the regularization allows $F((x, 0))$ to be any point on the line segment between $(2, 1) - (5, -2)$ (N.B. $(1, 0)$ is excluded since the x-axis is a set of measure $0$ here), while the original differential equation doesn't allow $\phi'(t)_x = 3$ for $t \in (1, \infty)$. A \kras solution that's not \flip is
  $$
  \phi(t) = \begin{cases}
    (2t, t-1) & t \in [0, 1] \\
    (t + 1, 0) & t \in (1, \infty)
  \end{cases}
  $$
  since $(1, 0) \not\in F((t+1, 0))$ (i.e. it doesn't lie on the line segment between $(2, 1) - (5, -2)$), but $(1, 0) \in F_K((t + 1, 0))$ since $F_k((\cdot, 0))$ is the triangle between $(2, 1), (5, -2), (1, 0)$.
\end{ex}
\begin{ex}{1.11}
  Start by putting
  $$
  \phi_{0}(t) = \begin{cases}
    (2t, t) & t \in [0, 2/3] \\
    (5(t-2/3) + 4/3, -2(t-2/3) + 2/3) & t \in (2/3, 1]
  \end{cases}
  $$
  Next we will define $\phi_i$ recursively. To start, for any  $i$ put $g_i : [0, 2] \to \mathbb{R}$ as
  $$
  g_i(t) = \begin{cases}
    \phi_i(t) & t \in [0, 1] \\
    \phi_i(t-1) + 3 & t \in (1, 2]
  \end{cases},
  $$
  i.e. duplicating $\phi_i$. Now put $\phi_{i+1}(t) = \frac{g_i(2t)}{2}$. Notably $i$ there're $2^i$ triangles between $\phi_x$ and $\phi_{x, i}$ with base $\sqrt{2}/2^i$ and height $h_0 / 2^i$ ($h_0$ being the height of the triangle formed by $\phi_0$) so that
  $$
  \int_0^1 \parens{\phi_x - \phi_{i,x}} = 2^i \cdot \frac{1}{2} \cdot \frac{\sqrt{2}}{2^i} \cdot \frac{h_0}{2^i} = \frac{h_0\sqrt{2}}{2^{i+1}}
  $$
  A similar argument can be made about $\phi_y$ and $\phi_{i, y}$ and thus $\phi_i \to \phi$ uniformly. Lastly, specify $e_i$ so that $e_{i,x} = 0$ and $e_{i,y}$ is so that $\phi_{i, y}$ is negative when $\dot{\phi}_y = 2$ and positive otherwise. This error indeed shrinks to $0$ as the heights of the triangles formed by $\phi_{i, y}, \phi_y$ vanish (i.e. we can take the error to be $-h_i + \frac{\chi_i(t)}{i}$ where $\chi_i = \pm 1$, depending on what slope is needed).

  --- \, \\
  For $\psi$ proceed as above, but instead of splitting the interval into $2/3$ and $1/3$, split it into $\lambda_1, \lambda_2, \lambda_3$ so that $\sum_i \lambda_i = 1$ and $2 \lambda_1 + 1\lambda_2 + 3\lambda_3 = 2$. Indeed $\psi_{i, y}$ will still vanish (as the only difference is the inclusion of subintervals with $0$ slope) and by the second equality as $i \to \infty$ $\psi_{i, x} \to 2t$.
\end{ex}
\begin{ex}{1.14}
  If the initial condition is in the top left or bottom right quadrants then put $\sigma(t) = 1$ until the solution hits the x-axis. Then put $\sigma(t) = 2$ until the solution hits the y-axis, followed by $\sigma(t) = 1$ until it hits the x-axis. Proceed this way and the solution will converge to the origin as everytime the solution hits either the x or y axis and switches it will get strictly closer to the origin until it reaches it.
\end{ex}
\end{document}
