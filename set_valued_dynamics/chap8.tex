\documentclass{article}

\title{Chapter 8 Exercises}
\usepackage{dztex}
\usepackage{cleveref}
\usepackage{amssymb} % \rightrightarrows

\fancyhead[L]{Set-Valued, Convex and Nonsmooth Analysis in Dynamics and Control \vspace{1mm}}

\newenvironment{ex}[1]
  {\renewcommand\theexercise{#1}\exercise}
  {\endexercise}

\newcommand{\B}{\mathbb{B}}
\newcommand{\ik}{i_k}
\DeclareMathOperator*{\argmin}{arg min}
\DeclareMathOperator*{\gph}{gph}
\DeclareMathOperator*{\dom}{dom}
\DeclareMathOperator*{\sgn}{sgn}
\DeclareMathOperator*{\spn}{span}
\DeclareMathOperator*{\inte}{int}
\DeclareMathOperator*{\con}{con}
\DeclareMathOperator*{\epi}{epi}
\newcommand{\clo}[1]{\overline{#1}}
\newcommand{\R}[1]{\mathbb{R}^{#1}}
\newcommand{\mr}{\mathcal{R}}
\newcommand{\idot}[2]{#1 \cdot #2}
\newcommand{\pdot}[2]{\parens{#1 \cdot #2}}

\begin{document}
\begin{ex}{8.5} %;
  Let $(x, y) \not\in \gph M$ so that $y \ne Qx$. Let $e$ be a unit eigen vector of $Q$ with corresponding eigen value $\lambda \ge 0$ (we know it's non-negative because $Q$ is positive). Fix $0 < \epsilon < \lambda^{-1} \abs{\iprod{e}{y-Qx}}$ (possible because $y \ne Qx$). Also put
  $$
  q = -\epsilon \sgn \iprod{e}{y - Qx} e
  $$
  Then we have
  $$
  \iprod{x - (x - q)}{y - Q(x - q)} = \iprod{q}{y - Mx} + \iprod{q}{Qq} = -\epsilon \abs{\iprod{e}{y - Qx}} + \epsilon^2 \lambda = \epsilon \parens{ \epsilon \lambda - \abs{\iprod{e}{y - Qx}} } < 0
  $$
  so that including $(x, y)$ in the graph breaks monotonicity, hence $M$ is maximal.
\end{ex} %:
\begin{ex}{8.11} %;
\end{ex} %:
\begin{ex}{8.12} %;
\end{ex} %:
\begin{ex}{8.16} %;
\end{ex} %:
\begin{ex}{8.18} %;
\end{ex} %:
\begin{ex}{8.19} %;
\end{ex} %:
\begin{ex}{8.20} %;
\end{ex} %:
\begin{ex}{8.22} %;
\end{ex} %:
\begin{ex}{8.24} %;
\end{ex} %:
\begin{ex}{8.25} %;
\end{ex} %:
\begin{ex}{8.26} %;
\end{ex} %:
\begin{ex}{8.27} %;
\end{ex} %:
\end{document}
