\documentclass{article}

\title{Chapter 5 Exercises}
\usepackage{dztex}
\usepackage{cleveref}
\usepackage{amssymb} % \rightrightarrows

\fancyhead[L]{Set-Valued, Convex and Nonsmooth Analysis in Dynamics and Control \vspace{1mm}}

\newenvironment{ex}[1]
  {\renewcommand\theexercise{#1}\exercise}
  {\endexercise}

\newcommand{\B}{\mathbb{B}}
\newcommand{\ik}{i_k}
\DeclareMathOperator*{\argmin}{arg min}
\DeclareMathOperator*{\gph}{gph}
\DeclareMathOperator*{\sgn}{sgn}
\DeclareMathOperator*{\spn}{span}
\DeclareMathOperator*{\inte}{Int}
\DeclareMathOperator*{\con}{con}
\newcommand{\clo}[1]{\overline{#1}}
\newcommand{\R}[1]{\mathbb{R}^{#1}}
\newcommand{\mr}{\mathcal{R}}
\newcommand{\idot}[2]{#1 \cdot #2}
\newcommand{\pdot}[2]{\parens{#1 \cdot #2}}

\begin{document}
\begin{ex}{5.1} %;
  First we want to show a lexographically minimal point exists. By compactness of $K$ it's bounded and closed. Projecting onto the first coordinate we get another bounded closed $1D$ set so that there's a minimal element. Call it $u_1$. Now consider $K_1 := K \cap \braces{ x_1 = u_1 }$, i.e. the intersection with the hyperplane specified by $x_1 = u_1$. This is closed since its the intersection of closed sets \& bounded because a subset of a bounded set, hence compact. Project $K_1$ onto its second coordinate to get another $1D$ compact set, hence with a minimal element $u_2$. We can iteratively do this $m$ times to come up with $u = (u_1, u_2, ..., u_m)$. By construction this is lexographically minimal. \, \\

  Now suppose there's a second point $w$ that's also lexographically minimal. By the definition of lexographic minimality of $u$ for some $i$ one of $u_i < w_i$. This means $w$ is not lexographically smaller than $u$, hence it's not lexographically minimal.
\end{ex} %:
\begin{ex}{5.3} %;
  The proof of this is very similar to Lemma 5.2. Put $A \subset [0, T]$ where $\dot{\phi}(t) \in F(\phi(t))$ for $t \in A$. Define $v(t) = \dot{\phi(t)}$ for $t \in A$ and for every $t \not\in A$ pick arbitrary $u^*(t) \in U(\phi(t))$ and  put $v(t) = f(\phi(t), u(t))$ then $v : [0, T] \to \R{n}$ where $v(t) = \dot{\phi}(t)$ a.e. $t$ and hence $v$ is measurable. Define $M : [0, T] \rightrightarrows U(\phi([0, T]))$ by
  $$
  M(t) = \begin{cases}
    \braces{ u \in U(\phi(t)) \mid v(t) = f(\phi(t), u) } & t \in A \\
    \braces{ u^*(t) } & t \in [0, T] \setminus A
  \end{cases}
  $$
  We have the following properties on $M$ at each $t \in A$:
  \begin{itemize}
    \item $M(t)$ is non-empty by construction of $v$
    \item Since $\phi([0, T])$ is bounded, $U$ is locally bounded, Exercise 2.12 tells us $U(\phi([0, T]))$ is bounded, hence $M(t)$ is bounded.
    \item We want to show each $M(t)$ is also closed so that $M(t)$ is compact. Let $u_i \in M(t)$ where $u_i \to u$. By definition of $M(t)$ we know $u_i \in U(\phi(t))$, and so by osc we know $u \in U(\phi(t))$ (our sequence in the domain is the constant sequence $\phi(t) \to \phi(t)$). Continuity of $f$ gives us $v(t) = f(\phi(t), u)$, so that $u \in M(t)$ and thus $M(t)$ is compact.
  \end{itemize}
  Trivially all the above conclusions follow for $t \not\in [0, T] \setminus A$, so that $M$ takes compact non-empty values. Just as in Lemma 5.2 define $u(t)$ as the lexographic minimal value in $M(t)$. We have $u : [0, t] \to U(\phi([0, T]))$ so that
  $$
  v(t) = f(\phi(t), u(t)) \; \forall t \in [0, T]
  $$
  All that's left to show is that $u$ is measurable. We do this just like in Lemma 5.2 inductively with Lusin's Theorem. We start again with $u_1, u_2, ... u_{k-1}$ coordinate functions measurable and will show $u_k$ is measurable. Pick $\epsilon > 0$. Lusin's Theorem gives us a compact $C \subset [0, T]$ so that $t \mapsto (u_1(t), u_2(t), ..., u_{k-1}(t), v(t))$ is continuous and $\mu([0, T] \setminus C) < \epsilon / 2$. Notably each $u_1, u_2, ..., u_{k-1}, v$ are continuous on $C$. \, \\

  Now the goal is to show for arbitrary $r$ the set
  $$
  S := \braces{ t \in C \mid u_k(t) \le r }
  $$
  is closed. If this is the case then we can show our claim the same way the end of Lemma 5.2 does after showing this set is closed. \, \\

  To this end, suppose $S$ is not closed, so that $\exists t_i \to \tau$ where $t_i \in S$. Since $C$ is closed and $t_i \in C$ we know $\tau \in C$. Because $U$ is locally bounded $u(t_i)$ is bounded and hence there's a convergent subsequence. Without relabeling we have $u(t_i) \to \clo{u}$, and by osc of $U$ we know $\clo{u} \in U(\tau)$. Continuity of $f, \phi$ tells us $f(\phi(t_i), u(t_i)) \to f(\phi(\tau), \clo{u})$ so that $\clo{u} \in M(\tau)$. \, \\

  Continuity of $u_1, u_2, ..., u_{k-1}$ on $C$ gives us $\clo{u}_i = u_i(\tau)$ for $i = 1, 2, ..., k-1$. However, by our initial assumption $u_k(\tau) > r$, and since $u(t_i)_k \le r \implies \clo{u}_k \le r$ so that
  $$
  u_k(\tau) > \clo{u}_k,
  $$
  but this contradicts the fact that $u(\tau)$ is lexographically minimal in $M(\tau)$ (since $\clo{u} \in M(\tau)$), hence indeed $S$ is closed.
\end{ex} %:
\begin{ex}{5.4} %;
  \begin{enumerate}
    \item[(a)]
      The forward implication comes from the definition of $\sup$. To prove the backward one, we start with $v$ so that $\idot{v}{p} \le \sup_{w \in C} \idot{w}{p}$ for every $p \in \R{n}$. Suppose for contradiction $v \not\in C$. By Theorem 4.16 that $\exists p \in \R{n}, \epsilon > 0$ so that
      $$
      \idot{w}{p} + \epsilon \le \idot{v}{p} \; \forall w \in C \implies \sup_{w\in C} \idot{w}{p} + \epsilon \le \idot{v}{p}
      $$
      which contradicts the initial assumption, so $v \in C$.
    \item[(b)]
      \begin{itemize}
        \item
          Let $(x_i, p_i) \to (x, p) \in C \times \R{n}$. Because $F$ is locally bounded $H(x_n, p_n)$ is finite. Thus, for each $n$, by definition of $\sup$ $\exists v_n \in F(x_n)$ where
          $$
          \sup_{v \in F(x_n)} \idot{v}{p_n} - \frac{1}{n} \le \idot{v_n}{p_n} \implies \limsup_{n \to \infty} H(x_n, p_n) \le \limsup_{n \to \infty} \idot{v_n}{p_n}
          $$
          Next, by definition of $\limsup$ $\exists \idot{v_{n_k}}{p_{n_k}} \to \limsup_{n \to \infty} \idot{v_n}{p_n}$. Since $F$ is locally bounded $\exists v_{n_{k_i}} \to \nu$, and by osc $\nu \in F(x)$. Using the fact that $p_n \to p$ we see $\idot{v_{n_k}}{p_{n_k}} \to \idot{\nu}{p}$ and so $\idot{\nu}{p} = \limsup_{n \to \infty} \idot{v_n}{p_n}$. Combined with the definition of $\sup$, the fact that $\nu \in F(x)$ means
          $$
          \limsup_{n \to \infty} \idot{v_n}{p_n} = \idot{\nu}{p} \le \sup_{v \in F(x)} \idot{v}{p} = H(x, p)
          $$
          Pulling in the initial major expression above we find our desired result
          $$
          \limsup_{n \to \infty} H(x_n, p_n) \le H(x, p)
          $$
        \item
          Let $p_n \to p \in C$. By the definition of $\sup$ we can construct, $v_n \in F(x)$ so that $\idot{v_n}{p} \to \sup_{v \in F(x)} \idot{v}{p}$. By osc and local boundedness of $F$ there's a convergent subsequence so that
          $$
          v_{n_i} \to \nu \in F(x) \implies \idot{v_n}{p} \to \idot{\nu}{p}
          $$
          We also have
          $$
          \idot{\nu}{p_n} \le \sup_{v \in F(x)} \idot{v}{p_n} \implies \liminf_n \idot{\nu}{p_n} \le \liminf_n \sup_{v \in F(x)} \idot{v}{p_n} = \liminf_n H(x, p_n)
          $$
          So that combining $\liminf_n \idot{\nu}{p_n} = \idot{\nu}{p} = H(x, p)$, and usc of $H$ our claim is shown since
          $$
          \limsup_n H(x, p_n) \le H(x, p) \le \liminf_n H(x, p_n) \implies H(x, p_n) \to H(x, p)
          $$
      \end{itemize}
  \end{enumerate}
\end{ex} %:
\begin{ex}{5.12} %;
\end{ex} %:
\begin{ex}{5.13} %;
\end{ex} %:
\begin{ex}{5.14} %;
\end{ex} %:
\begin{ex}{5.24} %;
\end{ex} %:
\begin{ex}{5.27} %;
\end{ex} %:
\end{document}
