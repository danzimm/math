\documentclass{article}

% dztex requires title to be specified
\title{Notes from Optimal Transportation Theory and Applications}

% common macros
\usepackage{dztex}

\newcommand{\push}[2]{#1_{\##2}}

\begin{document}
\subsection*{Introduction}

Previously, in undergrad, I worked with Dr. Marian Bocea. He helped me learn the functional analysis and measure theory needed to read a paper of his on optimal transportation theory (\cite{bocea_transport}). Recently I've picked my studies back up and have taken a liking to optimal transportation theory. After browsing a few pdfs online I decided to buy a book on the subject (\cite{opt_transport_theory_and_app}) to work out of. Below are small proofs that are left as exercises to the reader that throughout the text.

\subsection{Introduction to optimal transport theory}
This chapter was written by Fillipo Santambrogio. We will use the following definitions below:
\begin{itemize}
  \item $\mathcal{P}(Z)$ the set of probability measures on $Z$
  \item $\Omega \subset \mathbb{R}^m$
  \item $\mu, \nu \in \mathcal{P}(\Omega)$
  \item $\Pi(\mu, \nu) = \braces{\gamma \in \mathcal{P}(\Omega \times \Omega) \middle| \gamma(A \times \Omega) = \mu(A), \, \gamma(\Omega \times B) = \nu(B)\, \forall A, B \subset \Omega}$
  \item $c : \Omega \times \Omega \to [0, +\infty]$ symmetric and continuous
\end{itemize}
The Kantorovich problem is as follows:
\begin{equation}\label{eq:1}
  (K) \; \min \braces{\int_{\Omega \times \Omega} c \, \mathrm{d} \gamma \, \mid \, \gamma \in \Pi(\mu, \nu)}
\end{equation}
The following lemma comes from Remark 1:
\begin{lemma}\label{lem:1}
  If $\push{(id \times T)}{\mu} \in \Pi(\mu, \nu)$ for some $T : \Omega \to \Omega$ then
  \begin{equation}\label{eq:2}
    \nu = \push{T}{\mu} \iff \nu(A) = \mu(T^{-1}(A)) \textup{ for every measurable set } A
  \end{equation}
  and the functional in $(K)$ takes the form
  \begin{equation}\label{eq:3}
    \int_{\Omega \times \Omega} c(x, T(x)) \, \mathrm{d} \mu(x)
  \end{equation}
\end{lemma}
\begin{proof}
  Before beginning it's worth noting that $(id \times T)$ must be defined as $x \to (x, T(x))$ since otherwise pushing forward $\mu$ doesn't make sense. With that in mind, since $\push{(id \times T)}{\mu} \in \Pi(\mu, \nu)$ we have
  $$
  \nu(B) = \mu((id \times T)^{-1}(\Omega \times B)) = \mu(T^{-1}(B)) = \push{T}{\mu}(B)
  $$
  for every measurable $B$. The 2nd equality comes from the definition of the mapping as stated above. Additionally we have
  \begin{align*}
    \int_{\Omega \times \Omega} c \, \mathrm{d} \push{(id \times T)}{\mu}
    &= \int_{\Omega \times \Omega} c(x, y) \, \mathrm{d} \mu( (id \times T)^{-1} (x, y) ) \\
    &= \int_{\Omega \times \Omega} c(x, T(x)) \, \mathrm{d} \mu(x)
  \end{align*}
  where the last equality comes from the fact that
  $$
  (id \times T)^{-1} (A \times B) = \braces{ x \mid (x, T(x)) \in A \times B }
  $$
  i.e. $y$ is forced to $T(x)$ in order for the simple sets to have non-0 measure.
\end{proof}

\bibliographystyle{plain}
\bibliography{\jobname}
\end{document}
