\documentclass{article}

\title{Fractional Curvature Calculations}
\usepackage{dztex}
\usepackage{IEEEtrantools}
\DeclareMathOperator{\diam}{diam}
\DeclareMathOperator{\hm}{\mathcal{H}}
\DeclareMathOperator{\gam}{\Gamma}
\newcommand{\aeven}{\mathcal{A}_{\mathrm{even}}^+}
\newcommand{\aodd}{\mathcal{A}_{\mathrm{odd}}^+}
\newcommand{\bec}[1]{\mathbf{#1}}
\newcommand{\ud}{\mathrm{d}}

\usepackage{pgfplots}
\pgfplotsset{compat=1.5.1}
\pgfplotsset{every tick label/.append style={font=\footnotesize}}


\begin{document}

\subsection{Introduction} Here I will collect calculations done while exploring fractional curvature.

\subsection{$\kappa_\sigma$ of the unit circle}
We wish to compute
$$
\kappa_\sigma(z) := \parens{\int_{\aeven} - \int_{\aodd}} \frac{\parens{\bec{a} \cdot \bec{t}(z)} \bec{b} - \parens{\bec{b} \cdot \bec{t}(z)}\bec{a}}{r^{1+\sigma}} \, d\mathcal{H}^{2}\parens{\bec{a}, \bec{b}, r}
$$
for $C$ given by
$$
z(\phi) = \parens{\cos \phi, \sin \phi}, \phi \in [0, 2\pi].
$$
Due to symmetry $\kappa_\sigma(z(0)) = \kappa_\sigma(z(\phi)) \, \forall \phi \in (0, 2\pi]$, so we can focus on the case when $z = (1, 0)$. We have $\bec{t}(z) = (0, 1)$. 
in order to help us characterize $\aeven, \aodd$:
\begin{IEEEeqnarray*}{rCl}
  \prescript{1}{}{\aeven} &=& \braces{
    \parens{\begin{pmatrix} \cos\phi \\ \sin\phi\end{pmatrix}, \begin{pmatrix} -\sin\phi \\ \cos\phi \end{pmatrix}, r} \mid \phi \in \bracp{\frac{3\pi}{2}, 2\pi} \cup \bracp{0, \frac{\pi}{2}}, r \in [0, \infty)
  } \\
  \prescript{2}{}{\aeven} &=& \braces{
    \parens{\begin{pmatrix} \cos\phi \\ \sin\phi\end{pmatrix}, \begin{pmatrix} \sin\phi \\ -\cos\phi \end{pmatrix}, r} \mid \phi \in \bracp{\frac{\pi}{2}, \frac{3\pi}{2}}, r \in [0, \cos \parens{\pi - \phi})
  } \\
  \aeven &=& \prescript{1}{}{\aeven} \cup \prescript{2}{}{\aeven} \\
  \aodd &=& \braces{
    \parens{\begin{pmatrix} \cos\phi \\ \sin\phi\end{pmatrix}, \begin{pmatrix} \sin\phi \\ -\cos\phi \end{pmatrix}, r} \mid \phi \in \bracp{\frac{\pi}{2}, \frac{3\pi}{2}}, r \in [\cos \parens{\pi - \phi}, \infty)
  } \\
\end{IEEEeqnarray*}
These subsets are motivated by the following picture:
\begin{center}
\begin{tikzpicture}
\begin{axis}[ 
    axis lines = middle,
    ticks = none,
    axis line style={-},
    ymin=-1.1, ymax=1.1,
    xmin=-1.1, xmax=2,
    axis equal image,
    height=4.5in
]
\draw[color=white, fill = lightgray] (axis cs:1, -1.1) rectangle (axis cs:2, 1.1);

\draw[color=white, fill = blue, opacity=0.5] (axis cs:-1.1, -1.1) rectangle (axis cs:1, 1.1);
\begin{scope}[shift={(1, 0)}]
  \addplot[data cs=polar, red, domain=90:270,samples=180,smooth, fill=red, fill opacity=0.6] (x, {cos(x)});
\end{scope}

\node[label={{$\prescript{1}{}{\aeven}$}}] at (axis cs:1.5,-0.2) {};
\node[label={{$\prescript{2}{}{\aeven}$}}] at (axis cs:0.5,-0.3) {};
\node[label={{$\aodd$}}] at (axis cs:-0.5,0.3) {};

\draw (axis cs:0,0) circle [radius=1];
\node[label={225:{$z$}},circle,fill,inner sep=2pt] at (axis cs:1,0) {};

\draw (axis cs:0,0) -- (axis cs:{cos(30)},{sin(30)});
\draw (axis cs:1,0) -- (axis cs:{cos(30)},{sin(30)});
\draw[dashed] (axis cs:0,0) -- (axis cs:{cos(15)},{sin(15)});

\draw (axis cs:1.125,0.0) arc(0:{90+15}:{transformdirectionx(0.125)}) (axis cs:{1-0.125},0);
\node[label={45:{$\phi$}}] at (axis cs:1.06,0.06) {};
\end{axis}
\end{tikzpicture}
\end{center}
Before jumping into calculations observe that we can parameterize our subset of $\mathbb{R}^5$ via $(\theta, r)$, as shown in the definition of the subsets above and put 
$$
s(\theta) = \begin{cases}
  -1 & \theta \in [\pi/2, 3\pi/2] \\
  1 & \textup{otherwise}
\end{cases}.
$$
We can simplify our integrand as follows:
\begin{IEEEeqnarray*}{rCl}
  J(r, \theta) &=& \frac{\parens{\bec{a} \cdot \bec{t}(z)} \bec{b} - \parens{\bec{b} \cdot \bec{t}(z)}\bec{a}}{r^{1+\sigma}} \\
  &=& \frac{
    \parens{\begin{pmatrix} \cos\theta \\ \sin\theta \end{pmatrix} \cdot \begin{pmatrix} 0 \\ 1 \end{pmatrix}}s(\theta) \begin{pmatrix} -\sin \theta \\ \cos \theta \end{pmatrix}
    - \parens{s(\theta)\begin{pmatrix} -\sin\theta \\ \cos\theta \end{pmatrix} \cdot \begin{pmatrix} 0 \\ 1 \end{pmatrix}}s(\theta) \begin{pmatrix} \cos \theta \\ \sin \theta \end{pmatrix}
    }{r^{1+\sigma}} \\
    &=& \frac{s(\theta) \parens{\sin\theta \begin{pmatrix} -\sin \theta \\ \cos \theta \end{pmatrix} - \cos\theta \begin{pmatrix} \cos \theta \\ \sin \theta \end{pmatrix}}}{r^{1+\sigma}} = \frac{-s(\theta) \begin{pmatrix} \sin^2\theta + \cos^2\theta \\ -\sin\theta\cos\theta + \cos\theta\sin\theta \end{pmatrix}}{r^{1+\sigma}} \\
      &=& \begin{pmatrix} -1 \\ 0 \end{pmatrix} \frac{s(\theta)}{r^{1+\sigma}}
\end{IEEEeqnarray*}
Next we can start computing integrals, we begin by integrating over $\aeven$:
\begin{IEEEeqnarray*}{rCl}
  \int_{\aeven} \frac{s(\theta)}{r^{1+\sigma}} \, d\mathcal{H}^2(r, \theta) &=& \parens{\int_{3\pi/2}^{2\pi} + \int_{0}^{\pi/2}} \int_\epsilon^\infty \frac{s(\theta)}{r^{1+\sigma}} \, dr \, d\theta + \int_{\pi/2}^{3\pi/2} \int_{\epsilon}^{\cos(\pi-\theta)} \frac{s(\theta)}{r^{1+\sigma}} \, dr \, d\theta \\
  &=& \parens{\int_{3\pi/2}^{2\pi} + \int_{0}^{\pi/2}} \int_\epsilon^\infty \frac{1}{r^{1+\sigma}} \, d r \, d\theta - \int_{\pi/2}^{3\pi/2} \int_{\epsilon}^{\cos(\pi-\theta)} \frac{1}{r^{1+\sigma}} \, dr \, d\theta \\
  &=& -\frac{1}{\sigma}\parens{\int_{3\pi/2}^{2\pi} + \int_{0}^{\pi/2}} \parens{ 0 - \frac{1}{\epsilon^\sigma} } \, d\theta + \frac{1}{\sigma} \int_{\pi/2}^{3\pi/2} \parens{\frac{1}{\parens{\cos(\pi-\theta)}^\sigma} - \frac{1}{\epsilon^\sigma}} \, d \theta \\
  &=& \frac{\pi}{\sigma\epsilon^\sigma} - \frac{\pi}{\sigma\epsilon^\sigma} - \frac{1}{\sigma} \int_{\pi/2}^{-\pi/2} \parens{\sec \theta}^\sigma \, d\theta \\
  &=& \frac{1}{\sigma} \int_{-\pi/2}^{\pi/2} \parens{\sec \theta}^\sigma \, d \theta.
\end{IEEEeqnarray*}
Now for $\aodd$:
\begin{IEEEeqnarray*}{rCl}
  \int_{\aodd} \frac{s(\theta)}{r^{1+\sigma}} \, d\mathcal{H}^2(r, \theta) &=& \int_{\pi/2}^{3\pi/2} \int_{\cos(\pi - \theta)}^\infty \frac{s(\theta)}{r^{1+\sigma}} \, dr \, d\theta \\
  &=& - \int_{\pi/2}^{3\pi/2} \int_{\cos(\pi - \theta)}^\infty \frac{1}{r^{1+\sigma}} \, dr \, d\theta \\
  &=& \frac{1}{\sigma} \int_{\pi/2}^{3\pi/2} \parens{0 - \frac{1}{\parens{\cos(\pi - \theta)}^\sigma}} \, d\theta \\
  &=& \frac{1}{\sigma}\int_{\pi/2}^{-\pi/2} \parens{\sec \theta}^\sigma \, d \theta \\
  &=& -\frac{1}{\sigma}\int_{-\pi/2}^{\pi/2} \parens{\sec \theta}^\sigma \, d\theta.
\end{IEEEeqnarray*}
Putting these computations together we have:
\begin{IEEEeqnarray*}{rCl}
  \parens{\int_{\aeven} - \int_{\aodd}} \frac{\parens{\bec{a} \cdot \bec{t}(z)} \bec{b} - \parens{\bec{b} \cdot \bec{t}(z)}\bec{a}}{r^{1+\sigma}} \, d\mathcal{H}^{2}\parens{\bec{a}, \bec{b}, r} &=& \parens{\int_{\aeven} - \int_{\aodd}} \begin{pmatrix} -1 \\ 0 \end{pmatrix} \frac{s(\theta)}{r^{1+\sigma}} \, d\mathcal{H}^2 \\
    &=& \begin{pmatrix} -1 \\ 0 \end{pmatrix} \parens{\frac{1}{\sigma} \int_{-\pi/2}^{\pi/2} \parens{\sec \theta}^\sigma \, d\theta + \frac{1}{\sigma} \int_{-\pi/2}^{\pi/2} \parens{\sec \theta}^\sigma \, d\theta} \\
      &=& \begin{pmatrix} -1 \\ 0 \end{pmatrix} \frac{2}{\sigma} \int_{-\pi/2}^{\pi/2} \parens{\sec \theta}^\sigma \, d\theta.
\end{IEEEeqnarray*}
For now I don't have the know-how to evaluate this integral myself, but Wolfram Alpha tells me the following:
$$
\int_{-\pi/2}^{\pi/2} \parens{\sec \theta}^\sigma \, d\theta = \frac{\sqrt{\pi}\gam\parens{\frac{1}{2} - \frac{\sigma}{2}}}{\gam\parens{1-\frac{\sigma}{2}}}
$$
Thus we find:
\begin{IEEEeqnarray*}{rCl}
  \lim_{\sigma \uparrow 1} \frac{\parens{1 - \sigma}}{4} \kappa_\sigma &=& \lim_{\sigma\uparrow 1} \begin{pmatrix} -1 \\ 0 \end{pmatrix} \frac{2\parens{1-\sigma}\sqrt{\pi} \gam\parens{\frac{1}{2} - \frac{\sigma}{2}}}{4\sigma \gam\parens{1-\frac{\sigma}{2}}} \\
    &=& \begin{pmatrix} -1 \\ 0 \end{pmatrix} \frac{\sqrt{\pi}}{2} \lim_{\sigma \uparrow 1} \frac{\parens{1-\sigma}\gam\parens{\frac{1}{2} - \frac{\sigma}{2}}}{\sigma \gam\parens{1 - \frac{\sigma}{2}}} \\
      &=& \begin{pmatrix} -1 \\ 0 \end{pmatrix} \frac{1}{2} \lim_{\sigma\uparrow 1} \parens{1-\sigma} \gam\parens{\frac{1}{2} - \frac{\sigma}{2}} \\
      &=& \begin{pmatrix} -1 \\ 0 \end{pmatrix} = \kappa,
\end{IEEEeqnarray*}
i.e. $\kappa_\sigma$ converges to $\kappa$ with the factor of $\parens{1-\sigma}/4$ in front of it.
\end{document}
